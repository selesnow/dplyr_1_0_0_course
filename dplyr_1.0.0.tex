% Options for packages loaded elsewhere
\PassOptionsToPackage{unicode}{hyperref}
\PassOptionsToPackage{hyphens}{url}
%
\documentclass[
]{book}
\title{Введение в dplyr 1.0.0}
\author{Алексей Селезнёв}
\date{2022-01-04}

\usepackage{amsmath,amssymb}
\usepackage{lmodern}
\usepackage{iftex}
\ifPDFTeX
  \usepackage[T1]{fontenc}
  \usepackage[utf8]{inputenc}
  \usepackage{textcomp} % provide euro and other symbols
\else % if luatex or xetex
  \usepackage{unicode-math}
  \defaultfontfeatures{Scale=MatchLowercase}
  \defaultfontfeatures[\rmfamily]{Ligatures=TeX,Scale=1}
\fi
% Use upquote if available, for straight quotes in verbatim environments
\IfFileExists{upquote.sty}{\usepackage{upquote}}{}
\IfFileExists{microtype.sty}{% use microtype if available
  \usepackage[]{microtype}
  \UseMicrotypeSet[protrusion]{basicmath} % disable protrusion for tt fonts
}{}
\makeatletter
\@ifundefined{KOMAClassName}{% if non-KOMA class
  \IfFileExists{parskip.sty}{%
    \usepackage{parskip}
  }{% else
    \setlength{\parindent}{0pt}
    \setlength{\parskip}{6pt plus 2pt minus 1pt}}
}{% if KOMA class
  \KOMAoptions{parskip=half}}
\makeatother
\usepackage{xcolor}
\IfFileExists{xurl.sty}{\usepackage{xurl}}{} % add URL line breaks if available
\IfFileExists{bookmark.sty}{\usepackage{bookmark}}{\usepackage{hyperref}}
\hypersetup{
  pdftitle={Введение в dplyr 1.0.0},
  pdfauthor={Алексей Селезнёв},
  hidelinks,
  pdfcreator={LaTeX via pandoc}}
\urlstyle{same} % disable monospaced font for URLs
\usepackage{color}
\usepackage{fancyvrb}
\newcommand{\VerbBar}{|}
\newcommand{\VERB}{\Verb[commandchars=\\\{\}]}
\DefineVerbatimEnvironment{Highlighting}{Verbatim}{commandchars=\\\{\}}
% Add ',fontsize=\small' for more characters per line
\usepackage{framed}
\definecolor{shadecolor}{RGB}{248,248,248}
\newenvironment{Shaded}{\begin{snugshade}}{\end{snugshade}}
\newcommand{\AlertTok}[1]{\textcolor[rgb]{0.94,0.16,0.16}{#1}}
\newcommand{\AnnotationTok}[1]{\textcolor[rgb]{0.56,0.35,0.01}{\textbf{\textit{#1}}}}
\newcommand{\AttributeTok}[1]{\textcolor[rgb]{0.77,0.63,0.00}{#1}}
\newcommand{\BaseNTok}[1]{\textcolor[rgb]{0.00,0.00,0.81}{#1}}
\newcommand{\BuiltInTok}[1]{#1}
\newcommand{\CharTok}[1]{\textcolor[rgb]{0.31,0.60,0.02}{#1}}
\newcommand{\CommentTok}[1]{\textcolor[rgb]{0.56,0.35,0.01}{\textit{#1}}}
\newcommand{\CommentVarTok}[1]{\textcolor[rgb]{0.56,0.35,0.01}{\textbf{\textit{#1}}}}
\newcommand{\ConstantTok}[1]{\textcolor[rgb]{0.00,0.00,0.00}{#1}}
\newcommand{\ControlFlowTok}[1]{\textcolor[rgb]{0.13,0.29,0.53}{\textbf{#1}}}
\newcommand{\DataTypeTok}[1]{\textcolor[rgb]{0.13,0.29,0.53}{#1}}
\newcommand{\DecValTok}[1]{\textcolor[rgb]{0.00,0.00,0.81}{#1}}
\newcommand{\DocumentationTok}[1]{\textcolor[rgb]{0.56,0.35,0.01}{\textbf{\textit{#1}}}}
\newcommand{\ErrorTok}[1]{\textcolor[rgb]{0.64,0.00,0.00}{\textbf{#1}}}
\newcommand{\ExtensionTok}[1]{#1}
\newcommand{\FloatTok}[1]{\textcolor[rgb]{0.00,0.00,0.81}{#1}}
\newcommand{\FunctionTok}[1]{\textcolor[rgb]{0.00,0.00,0.00}{#1}}
\newcommand{\ImportTok}[1]{#1}
\newcommand{\InformationTok}[1]{\textcolor[rgb]{0.56,0.35,0.01}{\textbf{\textit{#1}}}}
\newcommand{\KeywordTok}[1]{\textcolor[rgb]{0.13,0.29,0.53}{\textbf{#1}}}
\newcommand{\NormalTok}[1]{#1}
\newcommand{\OperatorTok}[1]{\textcolor[rgb]{0.81,0.36,0.00}{\textbf{#1}}}
\newcommand{\OtherTok}[1]{\textcolor[rgb]{0.56,0.35,0.01}{#1}}
\newcommand{\PreprocessorTok}[1]{\textcolor[rgb]{0.56,0.35,0.01}{\textit{#1}}}
\newcommand{\RegionMarkerTok}[1]{#1}
\newcommand{\SpecialCharTok}[1]{\textcolor[rgb]{0.00,0.00,0.00}{#1}}
\newcommand{\SpecialStringTok}[1]{\textcolor[rgb]{0.31,0.60,0.02}{#1}}
\newcommand{\StringTok}[1]{\textcolor[rgb]{0.31,0.60,0.02}{#1}}
\newcommand{\VariableTok}[1]{\textcolor[rgb]{0.00,0.00,0.00}{#1}}
\newcommand{\VerbatimStringTok}[1]{\textcolor[rgb]{0.31,0.60,0.02}{#1}}
\newcommand{\WarningTok}[1]{\textcolor[rgb]{0.56,0.35,0.01}{\textbf{\textit{#1}}}}
\usepackage{longtable,booktabs,array}
\usepackage{calc} % for calculating minipage widths
% Correct order of tables after \paragraph or \subparagraph
\usepackage{etoolbox}
\makeatletter
\patchcmd\longtable{\par}{\if@noskipsec\mbox{}\fi\par}{}{}
\makeatother
% Allow footnotes in longtable head/foot
\IfFileExists{footnotehyper.sty}{\usepackage{footnotehyper}}{\usepackage{footnote}}
\makesavenoteenv{longtable}
\usepackage{graphicx}
\makeatletter
\def\maxwidth{\ifdim\Gin@nat@width>\linewidth\linewidth\else\Gin@nat@width\fi}
\def\maxheight{\ifdim\Gin@nat@height>\textheight\textheight\else\Gin@nat@height\fi}
\makeatother
% Scale images if necessary, so that they will not overflow the page
% margins by default, and it is still possible to overwrite the defaults
% using explicit options in \includegraphics[width, height, ...]{}
\setkeys{Gin}{width=\maxwidth,height=\maxheight,keepaspectratio}
% Set default figure placement to htbp
\makeatletter
\def\fps@figure{htbp}
\makeatother
\setlength{\emergencystretch}{3em} % prevent overfull lines
\providecommand{\tightlist}{%
  \setlength{\itemsep}{0pt}\setlength{\parskip}{0pt}}
\setcounter{secnumdepth}{5}
\usepackage{booktabs}
\ifLuaTeX
  \usepackage{selnolig}  % disable illegal ligatures
\fi
\usepackage[]{natbib}
\bibliographystyle{apalike}

\begin{document}
\maketitle

{
\setcounter{tocdepth}{1}
\tableofcontents
}
\hypertarget{ux432ux432ux435ux434ux435ux43dux438ux435}{%
\chapter*{Введение}\label{ux432ux432ux435ux434ux435ux43dux438ux435}}
\addcontentsline{toc}{chapter}{Введение}

\begin{center}\rule{0.5\linewidth}{0.5pt}\end{center}

\hypertarget{ux43e-ux43aux443ux440ux441ux435}{%
\section*{О курсе}\label{ux43e-ux43aux443ux440ux441ux435}}
\addcontentsline{toc}{section}{О курсе}

\texttt{dplyr} - один из наиболее популярных пакетов, который реализует грамматику манипуляции данными в языке R. Официально первый релиз \texttt{dplyr} был выпещен в 2014 году, но пакет развивался, и первая стабильная версия с устоявшимся синтаксисом, под номером 1.0.0 была выпущена весной 2020 года. Этому релизу предшествовала серия статей, в которых Хедли Викхем описывал все нововведения в \texttt{dplyr}. Этот мини курс появился из 5 видео уроков снятых по этим статьям.

В основе видео уроков лежат следующие статьи:

\begin{itemize}
\tightlist
\item
  \href{https://www.tidyverse.org/blog/2020/03/dplyr-1-0-0-select-rename-relocate/}{dplyr 1.0.0: select, rename, relocate}
\item
  \href{https://www.tidyverse.org/blog/2020/04/dplyr-1-0-0-colwise/}{dplyr 1.0.0: working across columns}
\item
  \href{https://www.tidyverse.org/blog/2020/04/dplyr-1-0-0-rowwise/}{dplyr 1.0.0: working within rows}
\item
  \href{https://www.tidyverse.org/blog/2020/03/dplyr-1-0-0-summarise/}{dplyr 1.0.0: new summarise() features}
\item
  \href{https://www.tidyverse.org/blog/2020/05/dplyr-1-0-0-last-minute-additions/}{dplyr 1.0.0: last minute additions}
\end{itemize}

\hypertarget{ux434ux43bux44f-ux43aux43eux433ux43e-ux44dux442ux43eux442-ux43aux443ux440ux441}{%
\section*{Для кого этот курс}\label{ux434ux43bux44f-ux43aux43eux433ux43e-ux44dux442ux43eux442-ux43aux443ux440ux441}}
\addcontentsline{toc}{section}{Для кого этот курс}

Для прохождения курса вы уже должны иметь навыки работы с инфраструктурой \texttt{tidyverse}. Приступать к прохождению курса ``Введение в dplyr 1.0.0'' я советую тем, кто уже имеет базовые навыки работы в R. Т.е. изначально я рекомендую вам пройти курс \href{https://selesnow.github.io/r4excel_users/}{``Язык R для пользователей Excel''}, и потом приступать к прохождению данного курса.

\hypertarget{ux440ux435ux43aux43eux43cux435ux43dux434ux430ux446ux438ux438-ux43fux43e-ux43fux440ux43eux445ux43eux436ux434ux435ux43dux438ux44e-ux43aux443ux440ux441ux430}{%
\section*{Рекомендации по прохождению курса}\label{ux440ux435ux43aux43eux43cux435ux43dux434ux430ux446ux438ux438-ux43fux43e-ux43fux440ux43eux445ux43eux436ux434ux435ux43dux438ux44e-ux43aux443ux440ux441ux430}}
\addcontentsline{toc}{section}{Рекомендации по прохождению курса}

Данный курс состоит из 5 видео уроков общей длительность 1 час 2 минуты. К каждому уроку есть рассмотренный в видео код, это сделано для вашего удобства, скопируйте его и выполняйте по мере просмотра видео лекции.

\hypertarget{ux43eux431-ux430ux432ux442ux43eux440ux435}{%
\section*{Об авторе}\label{ux43eux431-ux430ux432ux442ux43eux440ux435}}
\addcontentsline{toc}{section}{Об авторе}

Меня зовут Алексей Селезнёв, с 2008 года я являюсь практикующим аналитиком. На данный момент основной моей деятельностью является развитие отдела аналитики в агентстве интернет-маркетинга Netpeak.

Мною были разработаны такие R пакеты как: \texttt{ryandexdirect}, \texttt{rfacebookstat}, \texttt{timeperiodsR}, \texttt{rvkstat} и некоторые другие. На данный момент написанные мной пакеты только с CRAN были установленны более 130 000 раз.

Также я являюсь автором курса \href{https://needfordata.ru/r}{``Язык R для интернет-маркетинга''}.

Веду свой авторский \href{https://t.me/R4marketing}{Telegram} и \href{https://www.youtube.com/R4marketing/?sub_confirmation=1}{YouTube} канал R4marketing. Буду рад видеть вас в рядах подписчиков.

Периодически публикую статью на различных интернет медиа, зачастую это \href{https://habr.com/ru/users/selesnow/}{Хабр} и \href{https://netpeak.net/ru/blog/user/publication/826/}{Netpeak Journal}.

Неоднократно выступал на профильных конференциях по аналитике и интернет маркетингу, среди которых Матемаркетинг, GoAnalytics, Analyze, eCommerce, 8P и прочие.

Начиная с 2016 года всячески стараюсь популяризировать язык R среди русскоязычных аналитиков и маркетологов. Этот курс также был создан с этой целью.

\hypertarget{ux43fux440ux43eux433ux440ux430ux43cux43cux430-ux43aux443ux440ux441ux430}{%
\section*{Программа курса}\label{ux43fux440ux43eux433ux440ux430ux43cux43cux430-ux43aux443ux440ux441ux430}}
\addcontentsline{toc}{section}{Программа курса}

\begin{enumerate}
\def\labelenumi{\arabic{enumi}.}
\tightlist
\item
  \href{функции-select-rename_with-и-relocate.html}{Функции select(), rename\_with() и relocate()}
\item
  \href{функция-across.html}{Функция across()}
\item
  \href{перебор-строк-функцией-rowwise.html}{Перебор строк функцией rowwise()}
\item
  \href{обновлённая-функция-summarise.html}{Обновлённая функция summarise()}
\item
  \href{добавление-изменение-и-удаление-строк-дата-фрейма-через-rows_.html}{Добавление, изменение и удаление строк дата фрейма через rows\_*()}
\end{enumerate}

\hypertarget{ux431ux43bux430ux433ux43eux434ux430ux440ux43dux43eux441ux442ux438}{%
\section*{Благодарности}\label{ux431ux43bux430ux433ux43eux434ux430ux440ux43dux43eux441ux442ux438}}
\addcontentsline{toc}{section}{Благодарности}

Курс, и все сопутствующие материалы предоставляются бесплатно, но если у вас есть желание отблагодарить автора за этот видео курс вы можете перечислить любую произвольную сумму на \href{https://secure.wayforpay.com/payment/r4excel_users}{этой странице}.

Либо с помощью кнопки:

{Оплатить}

\hypertarget{ux444ux443ux43dux43aux446ux438ux438-select-rename_with-ux438-relocate}{%
\chapter{Функции select(), rename\_with() и relocate()}\label{ux444ux443ux43dux43aux446ux438ux438-select-rename_with-ux438-relocate}}

\hypertarget{ux43eux43fux438ux441ux430ux43dux438ux435}{%
\section{Описание}\label{ux43eux43fux438ux441ux430ux43dux438ux435}}

Первый урок курса посвящён таким операциям, как продвинутый выбор столбцов, их переименование и изменения порядка столбцов таблицы.

В этом видео уроке мы познакомимся с такими функциями как: \texttt{select()}, \texttt{rename\_with()}, \texttt{relocate()}, \texttt{any\_of()}, \texttt{all\_of()}.

В основе урока лежит статья \href{https://www.tidyverse.org/blog/2020/03/dplyr-1-0-0-select-rename-relocate/}{``dplyr 1.0.0: select, rename, relocate''}.

\hypertarget{ux432ux438ux434ux435ux43e}{%
\section{Видео}\label{ux432ux438ux434ux435ux43e}}

\hypertarget{ux43aux43eux434}{%
\section{Код}\label{ux43aux43eux434}}

\begin{Shaded}
\begin{Highlighting}[]
\CommentTok{\#devtools::install\_github("tidyverse/dplyr")}
\FunctionTok{library}\NormalTok{(dplyr, }\AttributeTok{warn.conflicts =} \ConstantTok{FALSE}\NormalTok{)}

\CommentTok{\# rename}
\CommentTok{\# Переименовать столбцы для устранение дублирования их имён}
\NormalTok{df1 }\OtherTok{\textless{}{-}} \FunctionTok{tibble}\NormalTok{(}\AttributeTok{a =} \DecValTok{1}\SpecialCharTok{:}\DecValTok{5}\NormalTok{, }\AttributeTok{a =} \DecValTok{5}\SpecialCharTok{:}\DecValTok{1}\NormalTok{, }\AttributeTok{.name\_repair =} \StringTok{"minimal"}\NormalTok{)}
\NormalTok{df1}

\NormalTok{df1 }\SpecialCharTok{\%\textgreater{}\%} \FunctionTok{rename}\NormalTok{(}\AttributeTok{b =} \DecValTok{2}\NormalTok{)}

\CommentTok{\# select}
\CommentTok{\# обращение к столбцам по типу}
\NormalTok{df2 }\OtherTok{\textless{}{-}} \FunctionTok{tibble}\NormalTok{(}\AttributeTok{x1 =} \DecValTok{1}\NormalTok{, }\AttributeTok{x2 =} \StringTok{"a"}\NormalTok{, }\AttributeTok{x3 =} \DecValTok{2}\NormalTok{, }\AttributeTok{y1 =} \StringTok{"b"}\NormalTok{, }\AttributeTok{y2 =} \DecValTok{3}\NormalTok{, }\AttributeTok{y3 =} \StringTok{"c"}\NormalTok{, }\AttributeTok{y4 =} \DecValTok{4}\NormalTok{)}

\CommentTok{\# числовые столбцы}
\NormalTok{df2 }\SpecialCharTok{\%\textgreater{}\%} \FunctionTok{select}\NormalTok{(is.numeric)}
\CommentTok{\# НЕ текстовые столбцы}
\NormalTok{df2 }\SpecialCharTok{\%\textgreater{}\%} \FunctionTok{select}\NormalTok{(}\SpecialCharTok{!}\NormalTok{is.character)}

\CommentTok{\# смешанный тип обращения}
\CommentTok{\# числовые стобцы, название которых начинается на X}
\NormalTok{df2 }\SpecialCharTok{\%\textgreater{}\%} \FunctionTok{select}\NormalTok{(}\FunctionTok{starts\_with}\NormalTok{(}\StringTok{"x"}\NormalTok{) }\SpecialCharTok{\&}\NormalTok{ is.numeric)}


\CommentTok{\# выбор полей с помощью функций any\_of и all\_of}
\NormalTok{vars }\OtherTok{\textless{}{-}} \FunctionTok{c}\NormalTok{(}\StringTok{"x1"}\NormalTok{, }\StringTok{"x2"}\NormalTok{, }\StringTok{"y1"}\NormalTok{, }\StringTok{"z"}\NormalTok{)}
\NormalTok{df2 }\SpecialCharTok{\%\textgreater{}\%} \FunctionTok{select}\NormalTok{(}\FunctionTok{any\_of}\NormalTok{(vars))}

\NormalTok{df2 }\SpecialCharTok{\%\textgreater{}\%} \FunctionTok{select}\NormalTok{(}\FunctionTok{all\_of}\NormalTok{(vars))}


\CommentTok{\# функция rename\_with}
\NormalTok{df2 }\SpecialCharTok{\%\textgreater{}\%} \FunctionTok{rename\_with}\NormalTok{(toupper)}

\NormalTok{df2 }\SpecialCharTok{\%\textgreater{}\%} \FunctionTok{rename\_with}\NormalTok{(toupper, }\FunctionTok{starts\_with}\NormalTok{(}\StringTok{"x"}\NormalTok{))}

\NormalTok{df2 }\SpecialCharTok{\%\textgreater{}\%} \FunctionTok{rename\_with}\NormalTok{(toupper, is.numeric)}


\CommentTok{\# relocate для изменения порядка стобцов}
\NormalTok{df3 }\OtherTok{\textless{}{-}} \FunctionTok{tibble}\NormalTok{(}\AttributeTok{w =} \DecValTok{0}\NormalTok{, }\AttributeTok{x =} \DecValTok{1}\NormalTok{, }\AttributeTok{y =} \StringTok{"a"}\NormalTok{, }\AttributeTok{z =} \StringTok{"b"}\NormalTok{)}
\CommentTok{\# переместить столбцы y, z в начало}
\NormalTok{df3 }\SpecialCharTok{\%\textgreater{}\%} \FunctionTok{relocate}\NormalTok{(y, z)}
\CommentTok{\# переместить текстовые столбцы вначало}
\NormalTok{df3 }\SpecialCharTok{\%\textgreater{}\%} \FunctionTok{relocate}\NormalTok{(is.character)}

\CommentTok{\# поместить столбец w после y}
\NormalTok{df3 }\SpecialCharTok{\%\textgreater{}\%} \FunctionTok{relocate}\NormalTok{(w, }\AttributeTok{.after =}\NormalTok{ y)}
\CommentTok{\# поместить столбец w перед y}
\NormalTok{df3 }\SpecialCharTok{\%\textgreater{}\%} \FunctionTok{relocate}\NormalTok{(w, }\AttributeTok{.before =}\NormalTok{ y)}
\CommentTok{\# переместить w в конец}
\NormalTok{df3 }\SpecialCharTok{\%\textgreater{}\%} \FunctionTok{relocate}\NormalTok{(w, }\AttributeTok{.after =} \FunctionTok{last\_col}\NormalTok{())}
\end{Highlighting}
\end{Shaded}

\hypertarget{ux444ux443ux43dux43aux446ux438ux44f-across}{%
\chapter{Функция across()}\label{ux444ux443ux43dux43aux446ux438ux44f-across}}

\hypertarget{ux43eux43fux438ux441ux430ux43dux438ux435-1}{%
\section{Описание}\label{ux43eux43fux438ux441ux430ux43dux438ux435-1}}

В этом уроке продемонстрирована работа с новой функцией \texttt{across()}, которая упрощает обращение к столбцам внутри таких функций как \texttt{mutate()} и \texttt{summarise()}. По сути функция \texttt{across()} заменяет функции с приставками \texttt{*\_at()} , \texttt{*\_if()}, \texttt{*\_all()}.

Обзор основан на статье Хедли Викхема \href{https://www.tidyverse.org/blog/2020/04/dplyr-1-0-0-colwise/}{``dplyr 1.0.0: working across columns''}.

\hypertarget{ux432ux438ux434ux435ux43e-1}{%
\section{Видео}\label{ux432ux438ux434ux435ux43e-1}}

\hypertarget{ux43aux43eux434-1}{%
\section{Код}\label{ux43aux43eux434-1}}

\begin{Shaded}
\begin{Highlighting}[]
\CommentTok{\# devtools::install\_github("tidyverse/dplyr")}
\FunctionTok{library}\NormalTok{(dplyr, }\AttributeTok{warn.conflicts =} \ConstantTok{FALSE}\NormalTok{)}

\CommentTok{\# тестовый дата фрейм}
\NormalTok{df }\OtherTok{\textless{}{-}} \FunctionTok{tibble}\NormalTok{(}\AttributeTok{g1 =} \FunctionTok{as.factor}\NormalTok{(}\FunctionTok{sample}\NormalTok{(letters[}\DecValTok{1}\SpecialCharTok{:}\DecValTok{4}\NormalTok{],}\AttributeTok{size =} \DecValTok{10}\NormalTok{, }\AttributeTok{replace =}\NormalTok{ T )),}
             \AttributeTok{g2 =} \FunctionTok{as.factor}\NormalTok{(}\FunctionTok{sample}\NormalTok{(LETTERS[}\DecValTok{1}\SpecialCharTok{:}\DecValTok{3}\NormalTok{],}\AttributeTok{size =} \DecValTok{10}\NormalTok{, }\AttributeTok{replace =}\NormalTok{ T )),}
             \AttributeTok{a  =} \FunctionTok{runif}\NormalTok{(}\DecValTok{10}\NormalTok{, }\DecValTok{1}\NormalTok{, }\DecValTok{10}\NormalTok{),}
             \AttributeTok{b  =} \FunctionTok{runif}\NormalTok{(}\DecValTok{10}\NormalTok{, }\DecValTok{10}\NormalTok{, }\DecValTok{20}\NormalTok{),}
             \AttributeTok{c  =} \FunctionTok{runif}\NormalTok{(}\DecValTok{10}\NormalTok{, }\DecValTok{15}\NormalTok{, }\DecValTok{30}\NormalTok{),}
             \AttributeTok{d  =} \FunctionTok{runif}\NormalTok{(}\DecValTok{10}\NormalTok{, }\DecValTok{1}\NormalTok{, }\DecValTok{50}\NormalTok{))}

\CommentTok{\# о чём пойдёт речь}
\DocumentationTok{\#\# копирование кода, когда требуется }
\DocumentationTok{\#\# произвести одну и туже операцию с разными функциями}
\NormalTok{df }\SpecialCharTok{\%\textgreater{}\%} 
  \FunctionTok{group\_by}\NormalTok{(g1, g2) }\SpecialCharTok{\%\textgreater{}\%} 
  \FunctionTok{summarise}\NormalTok{(}\AttributeTok{a =} \FunctionTok{mean}\NormalTok{(a), }\AttributeTok{b =} \FunctionTok{mean}\NormalTok{(b), }\AttributeTok{c =} \FunctionTok{mean}\NormalTok{(c), }\AttributeTok{d =} \FunctionTok{mean}\NormalTok{(c))}

\CommentTok{\# новый способ}
\DocumentationTok{\#\# теперь для таких преобразований можно}
\DocumentationTok{\#\# использовать тот же синтаксис что и в select()}
\DocumentationTok{\#\#\# посчитать среднее по столбцам от a до d}
\NormalTok{df }\SpecialCharTok{\%\textgreater{}\%} 
  \FunctionTok{group\_by}\NormalTok{(g1, g2) }\SpecialCharTok{\%\textgreater{}\%} 
  \FunctionTok{summarise}\NormalTok{(}\FunctionTok{across}\NormalTok{(a}\SpecialCharTok{:}\NormalTok{d, mean))}

\DocumentationTok{\#\#\# или посчитать среднее выбрав все числовые столбцы }
\NormalTok{df }\SpecialCharTok{\%\textgreater{}\%} 
  \FunctionTok{group\_by}\NormalTok{(g1, g2) }\SpecialCharTok{\%\textgreater{}\%} 
  \FunctionTok{summarise}\NormalTok{(}\FunctionTok{across}\NormalTok{(is.numeric, mean))}

\CommentTok{\# \#\#\#\#\#\#\#\#\#\#\#\#\#\#\#\#\#\#\#\#\#\#\#\#\#\#\#\#\#\#}
\CommentTok{\# Простой пример}
\CommentTok{\# аргументы функции accros}

\DocumentationTok{\#\# .cols {-} выбор столбцов по позиции, имени, функцией, типу данных, или комбинируя любые перечисленные способы}
\DocumentationTok{\#\# .fns {-} функция, или список функций которые необходимо применить к каждому столбцу}

\DocumentationTok{\#\# считаем количество униклаьных значений в текстовых полях}
\NormalTok{starwars }\SpecialCharTok{\%\textgreater{}\%} 
  \FunctionTok{summarise}\NormalTok{(}\FunctionTok{across}\NormalTok{(is.character, n\_distinct))}

\DocumentationTok{\#\# пример с фильтрацией данных}
\NormalTok{starwars }\SpecialCharTok{\%\textgreater{}\%} 
  \FunctionTok{group\_by}\NormalTok{(species) }\SpecialCharTok{\%\textgreater{}\%} 
  \FunctionTok{filter}\NormalTok{(}\FunctionTok{n}\NormalTok{() }\SpecialCharTok{\textgreater{}} \DecValTok{1}\NormalTok{) }\SpecialCharTok{\%\textgreater{}\%} 
  \FunctionTok{summarise}\NormalTok{(}\FunctionTok{across}\NormalTok{(}\FunctionTok{c}\NormalTok{(sex, gender, homeworld), n\_distinct))}

\DocumentationTok{\#\# комбинируем accross с другими вычислениями}
\NormalTok{starwars }\SpecialCharTok{\%\textgreater{}\%} 
  \FunctionTok{group\_by}\NormalTok{(homeworld) }\SpecialCharTok{\%\textgreater{}\%} 
  \FunctionTok{filter}\NormalTok{(}\FunctionTok{n}\NormalTok{() }\SpecialCharTok{\textgreater{}} \DecValTok{1}\NormalTok{) }\SpecialCharTok{\%\textgreater{}\%} 
  \FunctionTok{summarise}\NormalTok{(}\FunctionTok{across}\NormalTok{(is.numeric, mean, }\AttributeTok{na.rm =} \ConstantTok{TRUE}\NormalTok{), }
            \AttributeTok{n =} \FunctionTok{n}\NormalTok{())}

\CommentTok{\# \#\#\#\#\#\#\#\#\#\#\#\#\#\#\#\#\#\#\#\#\#\#\#\#\#\#\#\#\#\#}
\CommentTok{\# Чем accross лучше предыдущих функций с суфиксами \_at, \_if, \_all}

\DocumentationTok{\#\# 1. accross позволяет комбинировать различные вычисления внутри одной summarise }
\DocumentationTok{\#\# пример из статьи}
\NormalTok{df }\SpecialCharTok{\%\textgreater{}\%}
  \FunctionTok{group\_by}\NormalTok{(g1, g2) }\SpecialCharTok{\%\textgreater{}\%} 
  \FunctionTok{summarise}\NormalTok{(}
    \FunctionTok{across}\NormalTok{(is.numeric, mean), }
    \FunctionTok{across}\NormalTok{(is.factor, nlevels),}
    \AttributeTok{n =} \FunctionTok{n}\NormalTok{(), }
\NormalTok{  )}

\DocumentationTok{\#\# рабочий пример}
\NormalTok{starwars }\SpecialCharTok{\%\textgreater{}\%} 
  \FunctionTok{group\_by}\NormalTok{(species) }\SpecialCharTok{\%\textgreater{}\%} 
  \FunctionTok{summarise}\NormalTok{(}\FunctionTok{across}\NormalTok{(is.character, n\_distinct), }
            \FunctionTok{across}\NormalTok{(is.numeric, mean), }
            \FunctionTok{across}\NormalTok{(is.list, length), }
            \AttributeTok{n =} \FunctionTok{n}\NormalTok{()}
\NormalTok{  )}

\DocumentationTok{\#\# 2. уменьшает количество необходимых функций в dplyr, что облегчает их запоминание}
\DocumentationTok{\#\# 3. объединяет возможности функций с суфиксами if\_, at\_, и даёт возможность объединять их возможности}
\DocumentationTok{\#\# 4. accross не требует от вас использования функции vars для указания нужных столбцлв, как это было раньше}

\CommentTok{\# \#\#\#\#\#\#\#\#\#\#\#\#\#\#\#\#\#\#\#\#\#\#\#\#\#\#\#\#\#\#}
\CommentTok{\# перевод старого кода на accross}

\DocumentationTok{\#\# для перевода функций с суфиксами \_at, \_if, \_all используйте следующие правила}
\DocumentationTok{\#\#\# в accross первым агрументом будет:}
\DocumentationTok{\#\#\# Для *\_if() старый второй аргумент.}
\DocumentationTok{\#\#\# Для *\_at() старый второй аргумент с удаленным вызовом vars().}
\DocumentationTok{\#\#\# Для *\_all(), в качестве первого аргумента передайте функцию everything()}

\DocumentationTok{\#\# примеры}
\NormalTok{df }\OtherTok{\textless{}{-}} \FunctionTok{tibble}\NormalTok{(}\AttributeTok{y\_a  =} \FunctionTok{runif}\NormalTok{(}\DecValTok{10}\NormalTok{, }\DecValTok{1}\NormalTok{, }\DecValTok{10}\NormalTok{),}
             \AttributeTok{y\_b  =} \FunctionTok{runif}\NormalTok{(}\DecValTok{10}\NormalTok{, }\DecValTok{10}\NormalTok{, }\DecValTok{20}\NormalTok{),}
             \AttributeTok{x    =} \FunctionTok{runif}\NormalTok{(}\DecValTok{10}\NormalTok{, }\DecValTok{15}\NormalTok{, }\DecValTok{30}\NormalTok{),}
             \AttributeTok{d    =} \FunctionTok{runif}\NormalTok{(}\DecValTok{10}\NormalTok{, }\DecValTok{1}\NormalTok{, }\DecValTok{50}\NormalTok{))}

\DocumentationTok{\#\#\# из \_if в accross}
\NormalTok{df }\SpecialCharTok{\%\textgreater{}\%} \FunctionTok{mutate\_if}\NormalTok{(is.numeric, mean, }\AttributeTok{na.rm =} \ConstantTok{TRUE}\NormalTok{)}
\CommentTok{\# {-}\textgreater{}}
\NormalTok{df }\SpecialCharTok{\%\textgreater{}\%} \FunctionTok{mutate}\NormalTok{(}\FunctionTok{across}\NormalTok{(is.numeric, mean, }\AttributeTok{na.rm =} \ConstantTok{TRUE}\NormalTok{))}

\DocumentationTok{\#\#\# из \_at в accross}
\NormalTok{df }\SpecialCharTok{\%\textgreater{}\%} \FunctionTok{mutate\_at}\NormalTok{(}\FunctionTok{vars}\NormalTok{(}\FunctionTok{c}\NormalTok{(x, }\FunctionTok{starts\_with}\NormalTok{(}\StringTok{"y"}\NormalTok{))), mean, }\AttributeTok{na.rm =} \ConstantTok{TRUE}\NormalTok{)}
\CommentTok{\# {-}\textgreater{}}
\NormalTok{df }\SpecialCharTok{\%\textgreater{}\%} \FunctionTok{mutate}\NormalTok{(}\FunctionTok{across}\NormalTok{(}\FunctionTok{c}\NormalTok{(x, }\FunctionTok{starts\_with}\NormalTok{(}\StringTok{"y"}\NormalTok{)), mean, }\AttributeTok{na.rm =} \ConstantTok{TRUE}\NormalTok{))}

\DocumentationTok{\#\#\# из \_all в accroos}
\NormalTok{df }\SpecialCharTok{\%\textgreater{}\%} \FunctionTok{mutate\_all}\NormalTok{(mean, }\AttributeTok{na.rm =} \ConstantTok{TRUE}\NormalTok{)}
\CommentTok{\# {-}\textgreater{}}
\NormalTok{df }\SpecialCharTok{\%\textgreater{}\%} \FunctionTok{mutate}\NormalTok{(}\FunctionTok{across}\NormalTok{(}\FunctionTok{everything}\NormalTok{(), mean, }\AttributeTok{na.rm =} \ConstantTok{TRUE}\NormalTok{))}
\end{Highlighting}
\end{Shaded}

\hypertarget{ux43fux435ux440ux435ux431ux43eux440-ux441ux442ux440ux43eux43a-ux444ux443ux43dux43aux446ux438ux435ux439-rowwise}{%
\chapter{Перебор строк функцией rowwise()}\label{ux43fux435ux440ux435ux431ux43eux440-ux441ux442ux440ux43eux43a-ux444ux443ux43dux43aux446ux438ux435ux439-rowwise}}

\hypertarget{ux43eux43fux438ux441ux430ux43dux438ux435-2}{%
\section{Описание}\label{ux43eux43fux438ux441ux430ux43dux438ux435-2}}

В этом видео мы разберёмся с функцией \texttt{rowwise()}, из пакета \texttt{dplyr}.

Данная функция позволяет вам осуществить перебор строк таблицы не прибегая к циклам и функциям семейства \texttt{apply} или им подобным.

В основе урока лежит статья \href{https://www.tidyverse.org/blog/2020/04/dplyr-1-0-0-rowwise/}{``dplyr 1.0.0: working within rows''}.

\hypertarget{ux432ux438ux434ux435ux43e-2}{%
\section{Видео}\label{ux432ux438ux434ux435ux43e-2}}

\hypertarget{ux43aux43eux434-2}{%
\section{Код}\label{ux43aux43eux434-2}}

\begin{Shaded}
\begin{Highlighting}[]
\CommentTok{\#devtools::install\_github("tidyverse/dplyr")}
\FunctionTok{library}\NormalTok{(dplyr)}

\CommentTok{\# test data}
\NormalTok{df }\OtherTok{\textless{}{-}} \FunctionTok{tibble}\NormalTok{(}
  \AttributeTok{student\_id =} \DecValTok{1}\SpecialCharTok{:}\DecValTok{4}\NormalTok{, }
  \AttributeTok{test1 =} \DecValTok{10}\SpecialCharTok{:}\DecValTok{13}\NormalTok{, }
  \AttributeTok{test2 =} \DecValTok{20}\SpecialCharTok{:}\DecValTok{23}\NormalTok{, }
  \AttributeTok{test3 =} \DecValTok{30}\SpecialCharTok{:}\DecValTok{33}\NormalTok{, }
  \AttributeTok{test4 =} \DecValTok{40}\SpecialCharTok{:}\DecValTok{43}
\NormalTok{)}

\NormalTok{df}

\CommentTok{\# попытка посчитать среднюю оценку по студету}
\NormalTok{df }\SpecialCharTok{\%\textgreater{}\%} \FunctionTok{mutate}\NormalTok{(}\AttributeTok{avg =} \FunctionTok{mean}\NormalTok{(}\FunctionTok{c}\NormalTok{(test1, test2, test3, test4)))}

\CommentTok{\# используем rowwise для преобразования фрейма}
\NormalTok{rf }\OtherTok{\textless{}{-}} \FunctionTok{rowwise}\NormalTok{(df, student\_id)}
\NormalTok{rf}

\NormalTok{rf }\SpecialCharTok{\%\textgreater{}\%} \FunctionTok{mutate}\NormalTok{(}\AttributeTok{avg =} \FunctionTok{mean}\NormalTok{(}\FunctionTok{c}\NormalTok{(test1, test2, test3, test4)))}

\CommentTok{\# тоже самое с использованием c\_across}
\NormalTok{rf }\SpecialCharTok{\%\textgreater{}\%} \FunctionTok{mutate}\NormalTok{(}\AttributeTok{avg =} \FunctionTok{mean}\NormalTok{(}\FunctionTok{c\_across}\NormalTok{(}\FunctionTok{starts\_with}\NormalTok{(}\StringTok{"test"}\NormalTok{))))}

\CommentTok{\# \#\#\#\#\#\#\#\#\#\#\#\#\#\#\#\#\#\#\#\#\#\#\#\#\#\#\#}
\CommentTok{\# некоторые арифметические операции векторизированы по умолчанию}
\NormalTok{df }\SpecialCharTok{\%\textgreater{}\%} \FunctionTok{mutate}\NormalTok{(}\AttributeTok{total =}\NormalTok{ test1 }\SpecialCharTok{+}\NormalTok{ test2 }\SpecialCharTok{+}\NormalTok{ test3 }\SpecialCharTok{+}\NormalTok{ test4)}

\CommentTok{\# этот подход можно использовать для вычисления среднего}
\NormalTok{df }\SpecialCharTok{\%\textgreater{}\%} \FunctionTok{mutate}\NormalTok{(}\AttributeTok{avg =}\NormalTok{ (test1 }\SpecialCharTok{+}\NormalTok{ test2 }\SpecialCharTok{+}\NormalTok{ test3 }\SpecialCharTok{+}\NormalTok{ test4) }\SpecialCharTok{/} \DecValTok{4}\NormalTok{)}

\CommentTok{\# так же вы можете использовать некоторые специальные функции}
\CommentTok{\# для вычисления некоторых статистик}
\NormalTok{df }\SpecialCharTok{\%\textgreater{}\%} \FunctionTok{mutate}\NormalTok{(}
  \AttributeTok{min =} \FunctionTok{pmin}\NormalTok{(test1, test2, test3, test4), }
  \AttributeTok{max =} \FunctionTok{pmax}\NormalTok{(test1, test2, test3, test4), }
  \AttributeTok{string =} \FunctionTok{paste}\NormalTok{(test1, test2, test3, test4, }\AttributeTok{sep =} \StringTok{"{-}"}\NormalTok{)}
\NormalTok{)}
\CommentTok{\# все векторизированные функции будут работать быстрее чем rowwise}
\CommentTok{\# но rowwise позволяет векторизировать любую функцию}

\CommentTok{\# \#\#\#\#\#\#\#\#\#\#\#\#\#\#\#\#\#\#\#\#\#\#\#\#\#\#\#\#\#\#\#\#\#\#}
\CommentTok{\# работа со столбцами списками}
\NormalTok{df }\OtherTok{\textless{}{-}} \FunctionTok{tibble}\NormalTok{(}
  \AttributeTok{x =} \FunctionTok{list}\NormalTok{(}\DecValTok{1}\NormalTok{, }\DecValTok{2}\SpecialCharTok{:}\DecValTok{3}\NormalTok{, }\DecValTok{4}\SpecialCharTok{:}\DecValTok{6}\NormalTok{),}
  \AttributeTok{y =} \FunctionTok{list}\NormalTok{(}\ConstantTok{TRUE}\NormalTok{, }\DecValTok{1}\NormalTok{, }\StringTok{"a"}\NormalTok{),}
  \AttributeTok{z =} \FunctionTok{list}\NormalTok{(sum, mean, sd)}
\NormalTok{)}

\CommentTok{\# мы можем перебором обработать каждый список}
\NormalTok{df }\SpecialCharTok{\%\textgreater{}\%} 
  \FunctionTok{rowwise}\NormalTok{() }\SpecialCharTok{\%\textgreater{}\%} 
  \FunctionTok{summarise}\NormalTok{(}
    \AttributeTok{x\_length =} \FunctionTok{length}\NormalTok{(x),}
    \AttributeTok{y\_type =} \FunctionTok{typeof}\NormalTok{(y),}
    \AttributeTok{z\_call =} \FunctionTok{z}\NormalTok{(}\DecValTok{1}\SpecialCharTok{:}\DecValTok{5}\NormalTok{)}
\NormalTok{  )}

\CommentTok{\# \#\#\#\#\#\#\#\#\#\#\#\#\#\#\#\#\#\#\#\#\#\#\#\#\#\#\#\#\#\#\#\#\#\#}
\CommentTok{\# симуляция случайных данных}
\NormalTok{df }\OtherTok{\textless{}{-}} \FunctionTok{tribble}\NormalTok{(}
  \SpecialCharTok{\textasciitilde{}}\NormalTok{id, }\SpecialCharTok{\textasciitilde{}}\NormalTok{ n, }\SpecialCharTok{\textasciitilde{}}\NormalTok{ min, }\SpecialCharTok{\textasciitilde{}}\NormalTok{ max,}
  \DecValTok{1}\NormalTok{,   }\DecValTok{3}\NormalTok{,     }\DecValTok{0}\NormalTok{,     }\DecValTok{1}\NormalTok{,}
  \DecValTok{2}\NormalTok{,   }\DecValTok{2}\NormalTok{,    }\DecValTok{10}\NormalTok{,   }\DecValTok{100}\NormalTok{,}
  \DecValTok{3}\NormalTok{,   }\DecValTok{2}\NormalTok{,   }\DecValTok{100}\NormalTok{,  }\DecValTok{1000}\NormalTok{,}
\NormalTok{)}

\CommentTok{\# используем rowwise для симуляции данных}
\NormalTok{df }\SpecialCharTok{\%\textgreater{}\%}
  \FunctionTok{rowwise}\NormalTok{(id) }\SpecialCharTok{\%\textgreater{}\%}
  \FunctionTok{mutate}\NormalTok{(}\AttributeTok{data =} \FunctionTok{list}\NormalTok{(}\FunctionTok{runif}\NormalTok{(n, min, max)))}

\NormalTok{df }\SpecialCharTok{\%\textgreater{}\%}
  \FunctionTok{rowwise}\NormalTok{(id) }\SpecialCharTok{\%\textgreater{}\%}
  \FunctionTok{summarise}\NormalTok{(}\AttributeTok{x =} \FunctionTok{runif}\NormalTok{(n, min, max))}

\CommentTok{\# \#\#\#\#\#\#\#\#\#\#\#\#\#\#\#\#\#\#\#\#\#\#\#\#\#\#\#\#\#\#\#\#\#\#}
\CommentTok{\# функция nest\_by позволяет создавать столбцы списки}
\NormalTok{by\_cyl }\OtherTok{\textless{}{-}}\NormalTok{ mtcars }\SpecialCharTok{\%\textgreater{}\%} \FunctionTok{nest\_by}\NormalTok{(cyl)}
\NormalTok{by\_cyl}

\CommentTok{\# такой подход удобно использовать при построении линейной модели}
\CommentTok{\# используем mutate для подгонки моели под каждую группа данных}
\NormalTok{by\_cyl }\OtherTok{\textless{}{-}}\NormalTok{ by\_cyl }\SpecialCharTok{\%\textgreater{}\%} \FunctionTok{mutate}\NormalTok{(}\AttributeTok{model =} \FunctionTok{list}\NormalTok{(}\FunctionTok{lm}\NormalTok{(mpg }\SpecialCharTok{\textasciitilde{}}\NormalTok{ wt, }\AttributeTok{data =}\NormalTok{ data)))}
\NormalTok{by\_cyl}
\CommentTok{\# теперь с помощью summarise }
\CommentTok{\# можно извлекать сводки или коэфициенты построенной модели}
\NormalTok{by\_cyl }\SpecialCharTok{\%\textgreater{}\%} \FunctionTok{summarise}\NormalTok{(broom}\SpecialCharTok{::}\FunctionTok{glance}\NormalTok{(model))}
\NormalTok{by\_cyl }\SpecialCharTok{\%\textgreater{}\%} \FunctionTok{summarise}\NormalTok{(broom}\SpecialCharTok{::}\FunctionTok{tidy}\NormalTok{(model))}
\end{Highlighting}
\end{Shaded}

\hypertarget{ux43eux431ux43dux43eux432ux43bux451ux43dux43dux430ux44f-ux444ux443ux43dux43aux446ux438ux44f-summarise}{%
\chapter{Обновлённая функция summarise()}\label{ux43eux431ux43dux43eux432ux43bux451ux43dux43dux430ux44f-ux444ux443ux43dux43aux446ux438ux44f-summarise}}

\hypertarget{ux43eux43fux438ux441ux430ux43dux438ux435-3}{%
\section{Описание}\label{ux43eux43fux438ux441ux430ux43dux438ux435-3}}

В этом уроке мы рассмотрим новые возможности функции \texttt{summarise()}.

Урок основан на статье \href{https://www.tidyverse.org/blog/2020/03/dplyr-1-0-0-summarise/}{``dplyr 1.0.0: new summarise() features''}.

\hypertarget{ux432ux438ux434ux435ux43e-3}{%
\section{Видео}\label{ux432ux438ux434ux435ux43e-3}}

\hypertarget{ux43aux43eux434-3}{%
\section{Код}\label{ux43aux43eux434-3}}

\begin{Shaded}
\begin{Highlighting}[]
\CommentTok{\#devtools::install\_github("tidyverse/dplyr")}
\FunctionTok{library}\NormalTok{(dplyr)}

\CommentTok{\# Основные изменения}
\DocumentationTok{\#\# теперь sunnarise может вернуть}
\DocumentationTok{\#\#\# вектор любой длинны}
\DocumentationTok{\#\#\# дата фрейм любой размерности}

\CommentTok{\# \#\#\#\#\#\#\#\#\#\#\#\#\#\#\#\#\#\#\#\#\#\#\#\#\#\#\#\#\#\#\#\#\#\#\#\#\#\#\#\#\#\#\#\#\#\#\#\#\#\#\#\#\#\#\#}
\CommentTok{\# тестовые данные}
\CommentTok{\# \#\#\#\#\#\#\#\#\#\#\#\#\#\#\#\#\#\#\#\#\#\#\#\#\#\#\#\#\#\#\#\#\#\#\#\#\#\#\#\#\#\#\#\#\#\#\#\#\#\#\#\#\#\#\#}
\NormalTok{df }\OtherTok{\textless{}{-}} \FunctionTok{tibble}\NormalTok{(}
  \AttributeTok{grp =} \FunctionTok{rep}\NormalTok{(}\DecValTok{1}\SpecialCharTok{:}\DecValTok{2}\NormalTok{, }\AttributeTok{each =} \DecValTok{5}\NormalTok{), }
  \AttributeTok{x =} \FunctionTok{c}\NormalTok{(}\FunctionTok{rnorm}\NormalTok{(}\DecValTok{5}\NormalTok{, }\SpecialCharTok{{-}}\FloatTok{0.25}\NormalTok{, }\DecValTok{1}\NormalTok{), }\FunctionTok{rnorm}\NormalTok{(}\DecValTok{5}\NormalTok{, }\DecValTok{0}\NormalTok{, }\FloatTok{1.5}\NormalTok{)),}
  \AttributeTok{y =} \FunctionTok{c}\NormalTok{(}\FunctionTok{rnorm}\NormalTok{(}\DecValTok{5}\NormalTok{, }\FloatTok{0.25}\NormalTok{, }\DecValTok{1}\NormalTok{), }\FunctionTok{rnorm}\NormalTok{(}\DecValTok{5}\NormalTok{, }\DecValTok{0}\NormalTok{, }\FloatTok{0.5}\NormalTok{)),}
\NormalTok{)}

\NormalTok{df}

\CommentTok{\# получим минимальные и максимальные значения для каждой группы}
\CommentTok{\# и поместим эти значения в строки}
\NormalTok{df }\SpecialCharTok{\%\textgreater{}\%} 
  \FunctionTok{group\_by}\NormalTok{(grp) }\SpecialCharTok{\%\textgreater{}\%} 
  \FunctionTok{summarise}\NormalTok{(}\AttributeTok{rng =} \FunctionTok{range}\NormalTok{(x))}

\DocumentationTok{\#\# функция range возвращает вектор длинны 2}
\FunctionTok{range}\NormalTok{(df}\SpecialCharTok{$}\NormalTok{x)}
\DocumentationTok{\#\# но функция summarise разворачивает его, }
\DocumentationTok{\#\# приводя каждое из возвращаемых значений в новую строку}

\CommentTok{\# тоже самое, но для столбцов}
\NormalTok{df }\SpecialCharTok{\%\textgreater{}\%} 
  \FunctionTok{group\_by}\NormalTok{(grp) }\SpecialCharTok{\%\textgreater{}\%} 
  \FunctionTok{summarise}\NormalTok{(}\FunctionTok{tibble}\NormalTok{(}\AttributeTok{min =} \FunctionTok{min}\NormalTok{(x), }\AttributeTok{mean =} \FunctionTok{mean}\NormalTok{(x)))}

\CommentTok{\# \#\#\#\#\#\#\#\#\#\#\#\#\#\#\#\#\#\#\#\#\#\#\#\#\#\#\#\#\#\#\#\#\#\#\#\#\#\#\#\#\#\#\#\#\#\#\#\#\#\#\#\#\#\#\#}
\CommentTok{\# Расчёт квантилей}
\CommentTok{\# \#\#\#\#\#\#\#\#\#\#\#\#\#\#\#\#\#\#\#\#\#\#\#\#\#\#\#\#\#\#\#\#\#\#\#\#\#\#\#\#\#\#\#\#\#\#\#\#\#\#\#\#\#\#\#}
\NormalTok{df }\SpecialCharTok{\%\textgreater{}\%} 
  \FunctionTok{group\_by}\NormalTok{(grp) }\SpecialCharTok{\%\textgreater{}\%} 
  \FunctionTok{summarise}\NormalTok{(}\AttributeTok{x =} \FunctionTok{quantile}\NormalTok{(x, }\FunctionTok{c}\NormalTok{(}\FloatTok{0.25}\NormalTok{, }\FloatTok{0.5}\NormalTok{, }\FloatTok{0.75}\NormalTok{)), }\AttributeTok{q =} \FunctionTok{c}\NormalTok{(}\FloatTok{0.25}\NormalTok{, }\FloatTok{0.5}\NormalTok{, }\FloatTok{0.75}\NormalTok{))}

\CommentTok{\# можем избежать дублирования кода и написать функцию для вычисления квантиля}
\NormalTok{quibble }\OtherTok{\textless{}{-}} \ControlFlowTok{function}\NormalTok{(x, }\AttributeTok{q =} \FunctionTok{c}\NormalTok{(}\FloatTok{0.25}\NormalTok{, }\FloatTok{0.5}\NormalTok{, }\FloatTok{0.75}\NormalTok{)) \{}
  \FunctionTok{tibble}\NormalTok{(}\AttributeTok{x =} \FunctionTok{quantile}\NormalTok{(x, q), }\AttributeTok{q =}\NormalTok{ q)}
\NormalTok{\}}
\CommentTok{\# используем собственную функцию в summarise}
\NormalTok{df }\SpecialCharTok{\%\textgreater{}\%} 
  \FunctionTok{group\_by}\NormalTok{(grp) }\SpecialCharTok{\%\textgreater{}\%} 
  \FunctionTok{summarise}\NormalTok{(}\FunctionTok{quibble}\NormalTok{(x, }\FunctionTok{c}\NormalTok{(}\FloatTok{0.25}\NormalTok{, }\FloatTok{0.5}\NormalTok{, }\FloatTok{0.75}\NormalTok{)))}

\CommentTok{\# доработаем функцию таким образом }
\CommentTok{\# что бы названия столбца подтягивались из аргумена}
\NormalTok{quibble2 }\OtherTok{\textless{}{-}} \ControlFlowTok{function}\NormalTok{(x, }\AttributeTok{q =} \FunctionTok{c}\NormalTok{(}\FloatTok{0.25}\NormalTok{, }\FloatTok{0.5}\NormalTok{, }\FloatTok{0.75}\NormalTok{)) \{}
  \FunctionTok{tibble}\NormalTok{(}\StringTok{"\{\{ x \}\}"} \SpecialCharTok{:}\ErrorTok{=} \FunctionTok{quantile}\NormalTok{(x, q), }\StringTok{"\{\{ x \}\}\_q"} \SpecialCharTok{:}\ErrorTok{=}\NormalTok{ q)}
\NormalTok{\}}

\NormalTok{df }\SpecialCharTok{\%\textgreater{}\%} 
  \FunctionTok{group\_by}\NormalTok{(grp) }\SpecialCharTok{\%\textgreater{}\%} 
  \FunctionTok{summarise}\NormalTok{(}\FunctionTok{quibble2}\NormalTok{(x, }\FunctionTok{c}\NormalTok{(}\FloatTok{0.25}\NormalTok{, }\FloatTok{0.5}\NormalTok{, }\FloatTok{0.75}\NormalTok{)))}


\CommentTok{\# мы не присваивали имена новых столбцов внутри summarise}
\CommentTok{\# потому что если функция возвращает объект сложной стурктуры}
\CommentTok{\# мы получим вложенные дата фреймы}
\NormalTok{out }\OtherTok{\textless{}{-}}\NormalTok{ df }\SpecialCharTok{\%\textgreater{}\%} 
  \FunctionTok{group\_by}\NormalTok{(grp) }\SpecialCharTok{\%\textgreater{}\%} 
  \FunctionTok{summarise}\NormalTok{(}\AttributeTok{quantile =} \FunctionTok{quibble2}\NormalTok{(y, }\FunctionTok{c}\NormalTok{(}\FloatTok{0.25}\NormalTok{, }\FloatTok{0.75}\NormalTok{)))}

\FunctionTok{str}\NormalTok{(out)}

\CommentTok{\# обращаемся к вложенному фрейму}
\NormalTok{out}\SpecialCharTok{$}\NormalTok{y}

\CommentTok{\# или к его столбцам}
\CommentTok{\# по смыслу такая конструкция напоминает объяденённые имена стобцов в электронных таблицах}
\NormalTok{out}\SpecialCharTok{$}\NormalTok{quantile}\SpecialCharTok{$}\NormalTok{y\_q}

\CommentTok{\# summarise + rowwise}
\CommentTok{\# эта комбинация функций теперь может заменить purrr и apply}
\FunctionTok{tibble}\NormalTok{(}\AttributeTok{path =} \FunctionTok{dir}\NormalTok{(}\AttributeTok{pattern =} \StringTok{"}\SpecialCharTok{\textbackslash{}\textbackslash{}}\StringTok{.csv$"}\NormalTok{)) }\SpecialCharTok{\%\textgreater{}\%} 
  \FunctionTok{rowwise}\NormalTok{(path) }\SpecialCharTok{\%\textgreater{}\%} 
  \FunctionTok{summarise}\NormalTok{(readr}\SpecialCharTok{::}\FunctionTok{read\_csv}\NormalTok{(path))}

\CommentTok{\# \#\#\#\#\#\#\#\#\#\#\#\#\#\#\#\#\#\#\#\#\#\#\#\#\#\#\#\#\#\#\#\#\#\#\#\#\#\#\#\#\#\#\#\#\#\#\#\#\#\#\#\#\#\#\#}
\CommentTok{\# Предыдущие подходы}
\CommentTok{\# \#\#\#\#\#\#\#\#\#\#\#\#\#\#\#\#\#\#\#\#\#\#\#\#\#\#\#\#\#\#\#\#\#\#\#\#\#\#\#\#\#\#\#\#\#\#\#\#\#\#\#\#\#\#\#}
\CommentTok{\# вычисляем квантили}
\NormalTok{df }\SpecialCharTok{\%\textgreater{}\%} 
  \FunctionTok{group\_by}\NormalTok{(grp) }\SpecialCharTok{\%\textgreater{}\%} 
  \FunctionTok{summarise}\NormalTok{(}\AttributeTok{y =} \FunctionTok{list}\NormalTok{(}\FunctionTok{quibble}\NormalTok{(y, }\FunctionTok{c}\NormalTok{(}\FloatTok{0.25}\NormalTok{, }\FloatTok{0.75}\NormalTok{)))) }\SpecialCharTok{\%\textgreater{}\%} 
\NormalTok{  tidyr}\SpecialCharTok{::}\FunctionTok{unnest}\NormalTok{(y)}

\NormalTok{df }\SpecialCharTok{\%\textgreater{}\%} 
  \FunctionTok{group\_by}\NormalTok{(grp) }\SpecialCharTok{\%\textgreater{}\%} 
  \FunctionTok{do}\NormalTok{(}\FunctionTok{quibble}\NormalTok{(.}\SpecialCharTok{$}\NormalTok{y, }\FunctionTok{c}\NormalTok{(}\FloatTok{0.25}\NormalTok{, }\FloatTok{0.75}\NormalTok{)))}
\end{Highlighting}
\end{Shaded}

\hypertarget{ux434ux43eux431ux430ux432ux43bux435ux43dux438ux435-ux438ux437ux43cux435ux43dux435ux43dux438ux435-ux438-ux443ux434ux430ux43bux435ux43dux438ux435-ux441ux442ux440ux43eux43a-ux434ux430ux442ux430-ux444ux440ux435ux439ux43cux430-ux447ux435ux440ux435ux437-rows_}{%
\chapter{Добавление, изменение и удаление строк дата фрейма через rows\_*()}\label{ux434ux43eux431ux430ux432ux43bux435ux43dux438ux435-ux438ux437ux43cux435ux43dux435ux43dux438ux435-ux438-ux443ux434ux430ux43bux435ux43dux438ux435-ux441ux442ux440ux43eux43a-ux434ux430ux442ux430-ux444ux440ux435ux439ux43cux430-ux447ux435ux440ux435ux437-rows_}}

\hypertarget{ux43eux43fux438ux441ux430ux43dux438ux435-4}{%
\section{Описание}\label{ux43eux43fux438ux441ux430ux43dux438ux435-4}}

В SQL мы часто используем операции изменения данных, такие как \texttt{INSERT}, \texttt{UPDATE} и \texttt{DELETE}, так вот начиная с версии \texttt{dplyr} 1.0.0 в пакете появилось целое семейство функций которые реализуют эти операции с фреймами на языке R.

Функции которые будут рассмотрены в этом видео:
- \texttt{rows\_insert()}
- \texttt{rows\_update()}
- \texttt{rows\_upsert()}
- \texttt{rows\_patch()}
- \texttt{rows\_delete()}

Также мы разберёмся с новым аргументом функции \texttt{summarise()}, \texttt{.groups}, который позволяет изменять группировку данных после их агрегации.

В основе урока лежит статья \href{https://www.tidyverse.org/blog/2020/05/dplyr-1-0-0-last-minute-additions/}{``dplyr 1.0.0: last minute additions''}.

\hypertarget{ux432ux438ux434ux435ux43e-4}{%
\section{Видео}\label{ux432ux438ux434ux435ux43e-4}}

\hypertarget{ux43aux43eux434-4}{%
\section{Код}\label{ux43aux43eux434-4}}

\begin{Shaded}
\begin{Highlighting}[]
\CommentTok{\#devtools::install\_github("tidyverse/dplyr")}
\FunctionTok{library}\NormalTok{(dplyr)}

\CommentTok{\# summarise + .groups}
\NormalTok{starwars }\SpecialCharTok{\%\textgreater{}\%} 
  \FunctionTok{group\_by}\NormalTok{(homeworld, species) }\SpecialCharTok{\%\textgreater{}\%} 
  \FunctionTok{summarise}\NormalTok{(}\AttributeTok{n =} \FunctionTok{n}\NormalTok{())}

\DocumentationTok{\#\# аргумент .groups}
\DocumentationTok{\#\#\# .groups = "drop\_last" удалит последнюю группу}
\DocumentationTok{\#\#\# .groups = "drop" удалит всю группировку}
\DocumentationTok{\#\#\# .groups = "keep" созранит всю группировку}
\DocumentationTok{\#\#\# .groups = "rowwise" разобъёт фрейм на группы как rowwise()}

\CommentTok{\# rows\_*()}
\DocumentationTok{\#\# rows\_update(x, y) обновляет строки в таблице x найденные в таблице y}
\DocumentationTok{\#\# rows\_patch(x, y) работает аналогично rows\_update() но заменяет только NA}
\DocumentationTok{\#\# rows\_insert(x, y) добавляет строки в таблицу x из таблицы y}
\DocumentationTok{\#\# rows\_upsert(x, y) обновляет найденные строки в x и добавляет не найденные из таблицы y}
\DocumentationTok{\#\# rows\_delete(x, y) удаляет строки из x найденные в таблице y.}

\CommentTok{\# Создаём тестовые данные}
\NormalTok{df }\OtherTok{\textless{}{-}} \FunctionTok{tibble}\NormalTok{(}\AttributeTok{a =} \DecValTok{1}\SpecialCharTok{:}\DecValTok{3}\NormalTok{, }\AttributeTok{b =}\NormalTok{ letters[}\FunctionTok{c}\NormalTok{(}\DecValTok{1}\SpecialCharTok{:}\DecValTok{2}\NormalTok{, }\ConstantTok{NA}\NormalTok{)], }\AttributeTok{c =} \FloatTok{0.5} \SpecialCharTok{+} \DecValTok{0}\SpecialCharTok{:}\DecValTok{2}\NormalTok{)}
\NormalTok{df}

\NormalTok{new }\OtherTok{\textless{}{-}} \FunctionTok{tibble}\NormalTok{(}\AttributeTok{a =} \FunctionTok{c}\NormalTok{(}\DecValTok{4}\NormalTok{, }\DecValTok{5}\NormalTok{), }\AttributeTok{b =} \FunctionTok{c}\NormalTok{(}\StringTok{"d"}\NormalTok{, }\StringTok{"e"}\NormalTok{), }\AttributeTok{c =} \FunctionTok{c}\NormalTok{(}\FloatTok{3.5}\NormalTok{, }\FloatTok{4.5}\NormalTok{))}
\NormalTok{new}

\CommentTok{\# БАзовые примеры}
\DocumentationTok{\#\# добавляем новые строки}
\NormalTok{df }\SpecialCharTok{\%\textgreater{}\%} \FunctionTok{rows\_insert}\NormalTok{(new)}

\DocumentationTok{\#\# row\_insert вернёт ошибку если мы попытаемся добавить уже существующую строку}
\NormalTok{df }\SpecialCharTok{\%\textgreater{}\%} \FunctionTok{rows\_insert}\NormalTok{(}\FunctionTok{tibble}\NormalTok{(}\AttributeTok{a =} \DecValTok{3}\NormalTok{, }\AttributeTok{b =} \StringTok{"c"}\NormalTok{))}

\DocumentationTok{\#\# если вы хотите обновить существующее значение необходимо использовать row\_update}
\NormalTok{df }\SpecialCharTok{\%\textgreater{}\%} \FunctionTok{rows\_update}\NormalTok{(}\FunctionTok{tibble}\NormalTok{(}\AttributeTok{a =} \DecValTok{3}\NormalTok{, }\AttributeTok{b =} \StringTok{"c"}\NormalTok{))}

\DocumentationTok{\#\# но rows\_update вернёт ошибку если вы попытаетесь обновить несуществующее значание}
\NormalTok{df }\SpecialCharTok{\%\textgreater{}\%} \FunctionTok{rows\_update}\NormalTok{(}\FunctionTok{tibble}\NormalTok{(}\AttributeTok{a =} \DecValTok{4}\NormalTok{, }\AttributeTok{b =} \StringTok{"d"}\NormalTok{))}

\DocumentationTok{\#\# rows\_patch заполнит только пропущенные значения по ключу}
\NormalTok{df }\SpecialCharTok{\%\textgreater{}\%} 
  \FunctionTok{rows\_patch}\NormalTok{(}\FunctionTok{tibble}\NormalTok{(}\AttributeTok{a =} \DecValTok{2}\SpecialCharTok{:}\DecValTok{3}\NormalTok{, }\AttributeTok{b =} \StringTok{"B"}\NormalTok{))}

\DocumentationTok{\#\# rows\_upsert также вы можете добавлять новые и заменять существуюие значения }
\DocumentationTok{\#\# функцией rows\_upsert}
\NormalTok{df }\SpecialCharTok{\%\textgreater{}\%} 
  \FunctionTok{rows\_upsert}\NormalTok{(}\FunctionTok{tibble}\NormalTok{(}\AttributeTok{a =} \DecValTok{3}\NormalTok{, }\AttributeTok{b =} \StringTok{"c"}\NormalTok{)) }\SpecialCharTok{\%\textgreater{}\%} 
  \FunctionTok{rows\_upsert}\NormalTok{(}\FunctionTok{tibble}\NormalTok{(}\AttributeTok{a =} \DecValTok{4}\NormalTok{, }\AttributeTok{b =} \StringTok{"d"}\NormalTok{))}

\CommentTok{\# \#\#\#\#\#\#\#\#\#\#\#\#\#\#\#\#\#\#\#\#\#\#\#\#\#\#\#\#\#\#\#\#}
\CommentTok{\# РАЗБЕРЁМ Аргументы немного более подробно}
\FunctionTok{set.seed}\NormalTok{(}\DecValTok{555}\NormalTok{)}

\CommentTok{\# менеджеры}
\NormalTok{managers }\OtherTok{\textless{}{-}} \FunctionTok{c}\NormalTok{(}\StringTok{"Paul"}\NormalTok{, }\StringTok{"Alex"}\NormalTok{, }\StringTok{"Tim"}\NormalTok{, }\StringTok{"Bill"}\NormalTok{, }\StringTok{"John"}\NormalTok{)}
\CommentTok{\# товары}
\NormalTok{products }\OtherTok{\textless{}{-}} \FunctionTok{tibble}\NormalTok{(}\AttributeTok{name  =} \FunctionTok{paste0}\NormalTok{(}\StringTok{"product\_"}\NormalTok{, LETTERS), }
                   \AttributeTok{price =} \FunctionTok{round}\NormalTok{(}\FunctionTok{runif}\NormalTok{(}\AttributeTok{n =} \FunctionTok{length}\NormalTok{(LETTERS), }\DecValTok{100}\NormalTok{, }\DecValTok{400}\NormalTok{), }\DecValTok{0}\NormalTok{))}

\CommentTok{\# функция генерации купленных товаров}
\NormalTok{prod\_list }\OtherTok{\textless{}{-}} \ControlFlowTok{function}\NormalTok{(prod\_list, size\_min, size\_max) \{}
  
\NormalTok{  prod }\OtherTok{\textless{}{-}} \FunctionTok{tibble}\NormalTok{(}\AttributeTok{product =} \FunctionTok{sample}\NormalTok{(prod\_list, }
                                  \AttributeTok{size =} \FunctionTok{round}\NormalTok{(}\FunctionTok{runif}\NormalTok{(}\DecValTok{1}\NormalTok{, size\_min, size\_max), }\DecValTok{0}\NormalTok{) ,}
                                  \AttributeTok{replace =}\NormalTok{ F))}
    \FunctionTok{return}\NormalTok{(prod)}
\NormalTok{\}}


\CommentTok{\# генерируем продажи}
\NormalTok{sales }\OtherTok{\textless{}{-}} \FunctionTok{tibble}\NormalTok{(}\AttributeTok{id         =} \DecValTok{1}\SpecialCharTok{:}\DecValTok{200}\NormalTok{,}
                \AttributeTok{manager\_id =} \FunctionTok{sample}\NormalTok{(managers, }\AttributeTok{size =} \DecValTok{200}\NormalTok{, }\AttributeTok{replace =}\NormalTok{ T),}
                \AttributeTok{refund     =} \ConstantTok{FALSE}\NormalTok{,}
                \AttributeTok{refund\_sum =} \DecValTok{0}\NormalTok{)}

\CommentTok{\# генерируем списки купленных товаров}
\NormalTok{sale\_proucts }\OtherTok{\textless{}{-}}
\NormalTok{    sales }\SpecialCharTok{\%\textgreater{}\%}
      \FunctionTok{rowwise}\NormalTok{(id) }\SpecialCharTok{\%\textgreater{}\%}
      \FunctionTok{summarise}\NormalTok{(}\FunctionTok{prod\_list}\NormalTok{(products}\SpecialCharTok{$}\NormalTok{name, }\DecValTok{1}\NormalTok{, }\DecValTok{6}\NormalTok{)) }\SpecialCharTok{\%\textgreater{}\%}
      \FunctionTok{left\_join}\NormalTok{(products, }\AttributeTok{by =} \FunctionTok{c}\NormalTok{(}\StringTok{"product"} \OtherTok{=} \StringTok{"name"}\NormalTok{))}
  
\CommentTok{\# объединяем продажи с товарами}
\NormalTok{sales }\OtherTok{\textless{}{-}} \FunctionTok{left\_join}\NormalTok{(sales, sale\_proucts, }\AttributeTok{by =} \StringTok{"id"}\NormalTok{)}

\CommentTok{\# возвраты}
\NormalTok{refund }\OtherTok{\textless{}{-}} \FunctionTok{sample\_n}\NormalTok{(sales, }\DecValTok{25}\NormalTok{) }\SpecialCharTok{\%\textgreater{}\%}
          \FunctionTok{mutate}\NormalTok{( }\AttributeTok{refund =} \ConstantTok{TRUE}\NormalTok{,}
                  \AttributeTok{refund\_sum =}\NormalTok{ price }\SpecialCharTok{*} \FloatTok{0.9}\NormalTok{) }\SpecialCharTok{\%\textgreater{}\%}
          \FunctionTok{select}\NormalTok{(}\SpecialCharTok{{-}}\NormalTok{price, }\SpecialCharTok{{-}}\NormalTok{manager\_id) }

\CommentTok{\# отмечаем возвраты в таблице продаж}
\NormalTok{sales }\SpecialCharTok{\%\textgreater{}\%}
  \FunctionTok{rows\_update}\NormalTok{(refund)}

\CommentTok{\# аргумент by}
\NormalTok{result }\OtherTok{\textless{}{-}}
\NormalTok{  sales }\SpecialCharTok{\%\textgreater{}\%}
    \FunctionTok{rows\_update}\NormalTok{(refund, }\AttributeTok{by =} \FunctionTok{c}\NormalTok{(}\StringTok{"id"}\NormalTok{, }\StringTok{"product"}\NormalTok{))}
\end{Highlighting}
\end{Shaded}

\hypertarget{ux437ux430ux43aux43bux44eux447ux435ux43dux438ux435}{%
\chapter*{Заключение}\label{ux437ux430ux43aux43bux44eux447ux435ux43dux438ux435}}
\addcontentsline{toc}{chapter}{Заключение}

Надеюсь данный мини курс по введению в \texttt{dplyr\ 1.0.0} был вам полезен, данный релиз созержал действительно много полезных доработок и новых функций.

Буду рад видеть вас среди подписчиков моего \href{https://t.me/R4marketing}{Telegram} и \href{https://www.youtube.com/R4marketing/?sub_confirmation=1}{YouTube} канала.

\emph{Алексей Селезнёв}

\end{document}
